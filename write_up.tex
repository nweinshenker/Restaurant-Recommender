% TODO go back and rewrite the abstract
% TODO write equations for distance metric
% TODO write equations for majority metric
% TODO eager vs lazy learners

\documentclass{report}
\usepackage[utf8]{inputenc}
\usepackage{cite}
\usepackage{amsfonts}
\usepackage{amsmath}
\usepackage{algorithm, algpseudocode}
\usepackage{hyperref}
\usepackage{amssymb}

\title{Restaurant Recommendations using K Nearest Neighbors}
\date{December\\ 2018}
\author{Michael Siem \\ \href{mailto:siem@wisc.edu}{siem@wisc.edu}
	\and Nathan Weinshenker \\ \href{mailto:nweinshenker@wisc.edu}{nweinshenker@wisc.edu}
	\and Jack Long \\ \href{mailto:jlong25@wisc.edu}{jlong25@wisc.edu}}

\begin{document}

\maketitle

\section*{Abstract}
With the growth of ecommerce and internet, predictions of user preferences is a hot topic in towards Machine Learning community. Our goal is to use Yelp's open data set \href {https://www.yelp.com/dataset/challenge} {Yelp Dataset Challenge} to compare two different methods—KNN and SVD-- towards implementing collaborative filtering to similar user preferences.  Our projects goal is 

%\section*{Data  Extraction}
%Our data required a lot of preprocessing where we acquired a conglomeration of user, business, and review date. In each data set, we have: 

\section*{Background Information}

Recommendation systems are a vital tool for providing users with personalized suggestions for items such as movies, music, and restaurants.
Product reviews data is widely accessible on services such as Spotify, Amazon, and Yelp.
There are two approaches that can be used to make recommendations to users based on this data, memory based, and model based. This activity will be focused on the former and touch briefly on the latter.

Model based 
	
\subsection*{K Nearest Neighbors}
	
K Nearest Neighbors (k-NN) is a memory based collaborative filter that relies upon comparing features of a user or item to other users or items to receive a recommendation. 
k-NN is also a \textbf{non-parametric} learning algorithm. Since k-NN has little or no prior knowledge about the distribution data, k-NN  builds the model structure based on the training data. This allows the algorithm to be highly versatile and simple to perform with little to no information. [\textbf{include citation}

\subsubsection*{History of Nearest Neighbors}

In 1992, Paul Resnick and fellow researchers at the University of Minnesota first studied collaborative filtering systems under the GroupLens project. [\textbf{Include University of Minnesota slides}] In collaboration with developers at a company named Tapesty, GroupLens help build upon a query engine that could compare ratings. All of this came to fruition out of the need to handle the large amounts of email and messages posted to newsgroups. With the growth of the Internet, collaborative filters systems have developed grown in usage by big name tech companies like Netflix, Amazon, and Yahoo as predictive features help capture larger potential customers base.

\paragraph{Nearest Neighbor algorithm}


\subsubsection*{ Our K Nearest Neighbor Algorithm}
\begin{algorithm}
  \caption{K Nearest Neighbour}
  \begin{algorithmic}
  	\State Classify (X,Y,x) // X:training data, Y: class labels of X, x:unknown sample
	\For{ t = 1 to T}
	\State Compute the distance between $d(X, x)$
	\State Sort all the distances in ascending order
	\EndFor
	\State Find the set Z containing indices for the k smallest distances $d(X_{i},x)$
	\State return the set of k closest points 
  \end{algorithmic}
\end{algorithm}


\subsubsection*{The Distance Function}

Several methods can be used to evaluate the distances between the neighbors and the point of interest.
We will first look at the use of the Euclidean distance or the straight-line distance between the points.

\begin{equation}
d(x_{i},x_{j}) = \sqrt{(x_{i1} - x_{j1})^2 + (x_{i2} - x_{j2})^2 + ... + (x_{ip} - x_{jp})^2}
\end{equation}


\subsubsection*{Estimating the rating}
At it's core, k-NN is a simple classification method where little or no prior knowledge about the distribution of the data is needed. K-nearest neighbor classification was first developed to perform discriminant analysis when estimates are unknown or difficult to determine. \textbf{include citation}
	
		
Following determining the distance between test sample and training sample. 
With 1 nearest neighbor rule, the predicted class of test sample is set equal to the true class of its nearest neighbor, where is a nearest neighbor to x if the distance 
		
$d(m_{i},x) = min_{j}(d(m_{j},x))$

		
%\subsubsection**{SVD Decomposition}
%		
%SVD is a matrix factorization technique that is usually used to reduce the number of features of a data set by reducing space dimensions from N to K where K < N. \\
%		
%Example matrix:
%			\[
%     			M=
%  				\begin{bmatrix}
%    					1 & 3 & ? & 3 & 1 \\
%    					3 & 2 & 5 & 8 & 7 \\
%    					3 & ? & 5 & ? & ? \\
%    					9 & ? & 3 & 2 & 1
%  				\end{bmatrix}
%			\]
%			\textit{ where M is sparse (i.e., “with missing entries”, not “containing a lot of zeros”)}
%			
%			
%			\begin{enumerate}
%				\item U is an n $x$ n orthogonal matrix
%				\item V is an d $x$ d orthogonal matrix
%				\item S is an n $x$ d diagonal matrix with nonnegative entries, and with the diagonal entries
%sorted from high to low 
%			\end{enumerate}
%			
%			A matrix with 




\section*{Warm-up}

\begin{enumerate}



\item Draw Voronoi diagram
\end{enumerate}

\section*{Main Activity}

The main activity will be evaluating a KNN as a restaurant recommendation system using the Yelp dataset from \href{https://www.yelp.com/dataset}{https://www.yelp.com/dataset}.
The dataset originally contains 5,996,996 reviews for 188,593 businesses in 10 metropolitan areas.
We preprocessed the data set to narrow it down to 1 restaurant with 500 reviews for the activity, therefore we will be doing a user-user comparison.  
The activity will done using the Matlab script KNN.m.

\begin{enumerate}

\item Choosing K value

\item Using different distance functions

\item Evaluate different classification metrics

\end{enumerate}

\begin {thebibliography}{999}
\bibliographystyle{plain}
	\bibitem{1}
	Leif E. Peterson
	Scholarpeida
	\href{http://www.scholarpedia.org/article/K-nearest_neighbor}{Scholarpedia}
	
\end{thebibliography}


\end{document}


